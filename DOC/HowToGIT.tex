\documentclass[a4paper,11pt]{article}

\usepackage[francais]{babel}
\usepackage[utf8]{inputenc}  
\usepackage[T1]{fontenc}   
\usepackage{graphicx}
\usepackage[nottoc, notlof, notlot]{tocbibind}
\usepackage{color}
\usepackage{hyperref}
\usepackage{caption} 
\usepackage{float}
\usepackage{multirow}
\usepackage[paper=a4paper,margin=1cm]{geometry}
\usepackage{verbatim}
\usepackage{amstext}
\usepackage[toc,page]{appendix}
\usepackage{xcolor}
\usepackage{listings}

\lstdefinestyle{BashInputStyle}{
  language=bash,
  basicstyle=\small\sffamily,
  numbers=left,
  numberstyle=\tiny,
  numbersep=3pt,
  frame=tb,
  columns=fullflexible,
  backgroundcolor=\color{yellow!20},
  linewidth=0.9\linewidth,
  xleftmargin=0.1\linewidth
}

\lstdefinestyle{BashOutputStyle}{
  language=bash,
  basicstyle=\small\sffamily,
  numbers=left,
  numberstyle=\tiny,
  numbersep=3pt,
  frame=tb,
  columns=fullflexible,
  backgroundcolor=\color{lightgray},
  linewidth=0.9\linewidth,
  xleftmargin=0.1\linewidth
}

\captionsetup[table]{skip=10pt}

\hypersetup{
    colorlinks=true,
    linktoc=all,
    linkcolor=black,
}

\title{How To work with GIT}
\date{\today}
\author{B. Mouginot, N. Thiolliere}

% Début du document
\begin{document}

\maketitle
\newgeometry{top=1.5cm, bottom=1.5cm, left=2.5cm, right=2cm}

\tableofcontents
%    \listoffigures
%    \listoftables

\section{Introduction}

	This document explains minimum requirements needed for participating to FCCI. A lot of additionnal features are detailled in the following on ine documentation \url{https://git-scm.com/book/fr/v2}.

	It is assumed here the reader has already a github account.

\section{FCCI2017 group settings}

	\subsection{Join the FCCI2017 group}

	\subsection{Fork the repository}

	\subsection{Clone your fork locally}

	\subsection{Remote set up}

	\begin{lstlisting}[style=BashInputStyle]
	git add nom_des_fichiers
	\end{lstlisting}
	\bigskip

\section{Work with your fork}



\end{document}
